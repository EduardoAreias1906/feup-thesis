\chapter{Thesis Manifest} \label{chap:manifest}
%% early-stage scoping for alignment and feasibility; presents Context, Motivation, Problem, Research Questions, and Hypothesis in a concise, self-contained format.

\section{Context} \label{sec:se1}
%% situates the project in fully online K--12 schooling, states scale/setting, and frames why behavior patterns matter for learning progress.

Connections Academy, supported by Pearson Online and Blended Learning, operates fully online public schools serving K–12 students across 32 U.S. states and enrolling over 100,000 learners (Pearson, 2022). Independent national reviews repeatedly find that students in full-time virtual schools, on average, underperform peers in traditional brick-and-mortar public schools on standardized measures (NEPC, 2023; CREDO, 2015). These patterns make it important to understand what students actually do in virtual classrooms and how their behavioral routines relate to learning progress.

\section{Motivation} \label{sec:ae2}
%% explains why the topic matters now (practical and scientific relevance) and highlights the evidence gap the thesis will address.

Despite the scale of full-time online schooling, there is limited empirical detail about students’ fine-grained learning activities—how they navigate content, participate in live sessions, and interact with teachers and peers—and how these behaviors connect to outcomes (NEPC, 2023). Addressing this evidence gap can inform practice in large online school networks and help target support to students who need it most.

\section{Problem} \label{sec:ae3}
%% states the core research problem precisely—identify, describe, and relate online behavior patterns (virtual choreographies) to academic success.

This dissertation investigates virtual choreographies—recurring sequences of online learning actions—as a lens to characterize K–12 students’ behavior at scale and to examine how such patterns are associated with academic success. Concretely, the aim is to identify, describe, and relate behavior patterns observed in activity logs to indicators of performance in fully online schools.

\section{Research Questions} \label{sec:ae4}
%% lists the guiding questions that structure the study design (discovery, variation, outcomes, and methodological approach).

\begin{enumerate}
  \item[\textbf{RQ1}] \textbf{Identification.} Which \emph{virtual choreographies} (recurring sequences of online learning actions) can be reliably discovered in Connections Academy data, and how stable are they over time?

  \item[\textbf{RQ2}] \textbf{Robustness across contexts.} To what extent do the discovered choreographies replicate across grades, subjects, schools, and cohorts (i.e., are they consistent patterns rather than school- or cohort-specific artifacts)?

  \item[\textbf{RQ3}] \textbf{Association with outcomes.} How are choreography \emph{attributes}—such as temporal regularity, balance between synchronous and asynchronous participation, frequency of teacher–student feedback, and navigation efficiency—associated with indicators of student success (e.g., progress, grades, completion), when controlling for available background variables?

  \item[\textbf{RQ4}] \textbf{Predictive utility (early signals).} Do choreography-based representations improve early prediction of student success over standard activity metrics, and how early can such signals be detected with useful accuracy?

  \item[\textbf{RQ5}] \textbf{Actionability and interpretability.} Which components within the choreographies (specific actions, transitions, or rhythms) most contribute to positive or negative outcomes, and how can these be translated into simple indicators to support teachers’ decisions?
\end{enumerate}


\section{Hypotheses} \label{sec:ae5}
%% Purpose: define "virtual choreography" and state testable expectations for the thesis.

\paragraph{Definition (virtual choreography).}
A \emph{virtual choreography} is the structured set of actions carried out among agents (e.g., student, teacher, platform tools) that unfolds over time and within a specific learning space (LMS, live sessions, communication channels).

\medskip
\noindent\textbf{H1 — Regularity.}
Choreographies exhibiting stable temporal rhythms (e.g., consistent study times and revisits across the week) are positively associated with academic success.

\noindent\textbf{H2 — Balanced participation.}
Choreographies that balance synchronous participation (live sessions) and asynchronous work (content study, assignment submission) relate to better outcomes than predominantly one-sided patterns.

\noindent\textbf{H3 — Feedback-centric interaction.}
Choreographies with frequent teacher–student feedback exchanges (asking questions, reading feedback, revising submissions) are positively associated with success.

\noindent\textbf{H4 — Purposeful navigation.}
Choreographies characterized by efficient, task-aligned navigation (timely access to resources, limited detours) are associated with stronger progress than fragmented, sporadic engagement.

\noindent\textbf{H5 — Risk signals (inverse).}
Choreographies marked by long inactivity gaps, last-minute bursts, or minimal interaction with teachers/peers correlate with weaker outcomes.

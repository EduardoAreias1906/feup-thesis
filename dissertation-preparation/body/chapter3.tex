\chapter{Proposed Approach}
\label{chap:approach}

\section{Overview}
Building upon the research context defined in the Thesis Manifest (Chapter~\ref{chap:manifest}) and the technical state of the art (Chapter~\ref{chap:stateofart}), this chapter details the methodological approach adopted to identify and analyze Virtual Choreographies in K--12 online schools. 

The proposed solution is a data-driven pipeline designed to transform raw interaction logs from the Connections Academy Learning Management System (LMS) into interpretable behavioral patterns. The approach combines the semantic abstraction of the \textit{Virtual Choreographies} framework~\cite{cassola_using_2022} with the unsupervised machine learning techniques reviewed in Chapter~\ref{chap:stateofart}.

The ultimate goal is to move from low-level clickstream data to high-level educational insights that allow us to distinguish successful learning routines from those that signal disengagement, with a particular focus on the ``at-risk'' populations identified in the literature~\cite{beck_atrisk_2024}.

\section{Methodological Architecture}
The research will follow a four-stage pipeline: (1) Data Collection and Preprocessing, (2) Semantic Abstraction, (3) Pattern Discovery (Clustering), and (4) Evaluation and Association.

\subsection{Stage 1: Data Collection and Context}
The dataset for this dissertation is provided by Pearson Online and Blended Learning, encompassing activity logs from 32 Connections Academy schools across the United States.
\begin{itemize}
    \item \textbf{Population:} The study focuses on K--12 students. To address the concerns raised by Beck~\cite{beck_prepared_2023}, the data processing will preserve metadata that allows us to distinguish between general enrollment and students identified as ``at-risk'' (due to prior academic failure or medical/social challenges).
    \item \textbf{Data Granularity:} The raw data consists of server-level logs capturing timestamped events (e.g., login times, page accesses, quiz submissions, live lesson attendance).
    \item \textbf{Privacy:} All data is pseudonymized to ensure student privacy. Personal identifiers are replaced with unique hash keys before the analysis begins.
\end{itemize}

\subsection{Stage 2: Semantic Abstraction (The Virtual Choreography)}
A core challenge identified in Section~\ref{sec:modeling} is the ``spaghetti model'' problem inherent in analyzing flexible online learning environments~\cite{bogarin_survey_2018}. To overcome this, we apply the \textit{Virtual Choreography} abstraction layer.

We will map raw URLs and system events into a finite set of \textbf{Semantic Actions}. As emphasized in the literature review regarding student isolation~\cite{fostering_curtis_2015}, this mapping will prioritize not only content consumption but also \textbf{social and interactive behaviors}.

Table~\ref{tab:mapping} illustrates proposed examples of this mapping strategy:

\begin{table}[h]
\centering
\begin{tabular}{|l|l|l|}
\hline
\textbf{Raw Log Pattern (Example)} & \textbf{Semantic Action} & \textbf{Category} \\ \hline
\texttt{URL:/math/algebra/quiz/start} & \textbf{Assessment\_Start} & Evaluation \\ \hline
\texttt{URL:/live-lesson/room-101} & \textbf{Synchronous\_Participation} & Social/Instruction \\ \hline
\texttt{URL:/mail/compose/teacher} & \textbf{Teacher\_Interaction} & Social \\ \hline
\texttt{URL:/feedback/view} & \textbf{Feedback\_Review} & Metacognition \\ \hline
\texttt{URL:/content/pdf/view} & \textbf{Content\_Study} & Asynchronous \\ \hline
\end{tabular}
\caption{Examples of mapping raw logs to Semantic Actions.}
\label{tab:mapping}
\end{table}

This abstraction creates a sequence of events $S = \{a_1, a_2, ..., a_n\}$ for each student per day, which constitutes the basis of a daily choreography.

\subsection{Stage 3: Pattern Discovery}
To answer \textbf{RQ1 (Identification)}, we will employ unsupervised learning to group similar daily sequences.
\begin{enumerate}
    \item \textbf{Vectorization:} Student daily activities will be converted into numerical vectors (e.g., using n-gram frequency or sequence embeddings) to represent the ``rhythm'' of their day.
    \item \textbf{Clustering:} As discussed in Section~\ref{sec:algorithms}, we will apply the \textbf{k-Means} algorithm to identify distinct clusters of behavior. The optimal number of clusters ($k$) will be determined using the Elbow Method and Silhouette Analysis.
    \item \textbf{Visualization:} To inspect the robustness of these clusters (\textbf{RQ2}), we will use \textbf{UMAP} (Uniform Manifold Approximation and Projection) to project the high-dimensional data into 2D space. This will allow for visual inspection of the separation between behavioral groups, ensuring the identified choreographies are distinct and interpretable.
\end{enumerate}

\section{Experimental Process and Evaluation Design}
The evaluation is designed to directly address the Research Questions and test the Hypotheses (H1--H5) defined in the Thesis Manifest.

\subsection{Stability and Robustness (Addressing RQ1 \& RQ2)}
Once clusters are identified, we will validate them by:
\begin{itemize}
    \item \textbf{Temporal Stability:} Analyzing if students remain in the same behavioral cluster over weeks or if they transition between ``types'' of learners (e.g., does a student transition from a ``Steady Worker'' to a ``Fragmented'' profile?).
    \item \textbf{Cross-Context Validation:} Checking if the same choreographies emerge across different grades (e.g., 5th grade vs.\ 10th grade) to ensure the patterns are universal behavioral habits rather than course-specific artifacts.
\end{itemize}

\subsection{Association with Outcomes (Addressing RQ3)}
To validate H1 (Regularity) and H2 (Balanced Participation), we will perform statistical correlation analysis between the identified choreographies and student success metrics.

Given the specific needs of the at-risk population discussed in Section~\ref{sec:at-risk}, ``Success'' will be defined by two complementary metrics:
\begin{enumerate}
    \item \textbf{Academic Performance:} Final course grades and standardized test scores.
    \item \textbf{Retention:} The student's ability to maintain enrollment and avoid dropout. As noted by Beck~\cite{beck_atrisk_2024}, for many at-risk students, retention is a critical indicator of educational success.
\end{enumerate}

We will use regression models to quantify the impact of adopting a specific choreography on these outcomes, controlling for demographic background.

\subsection{Predictive Utility (Addressing RQ4)}
We will train supervised classification models (specifically Random Forests, due to their interpretability) to predict student success at different snapshots of the semester (e.g., Week 4, Week 8). We will compare two conditions:
\begin{enumerate}
    \item \textbf{Baseline:} Prediction using only standard metrics (login count, total time online).
    \item \textbf{Experimental:} Prediction using standard metrics + \textbf{Choreography Cluster ID}.
\end{enumerate}
This will determine if the ``shape'' of behavior adds predictive power beyond the simple volume of activity.

\section{Expected Results}
We anticipate identifying 4--6 distinct behavioral choreographies. Based on the hypotheses presented in Chapter~\ref{chap:manifest}:
\begin{itemize}
    \item We expect to confirm that \textbf{Regularity} (H1) is a stronger predictor of success than total time spent.
    \item We anticipate uncovering a ``risk'' profile characterized by high asynchronous activity but low synchronous interaction (validating H5), supporting the concerns about isolation raised by Curtis and Werth~\cite{fostering_curtis_2015}.
    \item The results will provide a new layer of analytics for Connections Academy, enabling teachers to intervene based on \textit{how} a student is learning, not just \textit{if} they are logging in.
\end{itemize}

\section{Work Plan}
The following timeline details the schedule for the completion of the dissertation, ensuring all deliverables are met by the final deadline in June/July 2026.

\begin{table}[h]
\centering
\begin{tabular}{|l|p{9cm}|}
\hline
\textbf{Month} & \textbf{Activity} \\ \hline
\textbf{February} & \textbf{Data Preprocessing \& Abstraction:} Cleaning the Connections Academy dataset; implementing the mapping from URLs to Semantic Actions; separating cohorts. \\ \hline
\textbf{March} & \textbf{Clustering \& Discovery (RQ1, RQ2):} Running k-Means and UMAP; refining the number of clusters; characterizing the identified choreographies. \\ \hline
\textbf{April} & \textbf{Association Analysis (RQ3):} Statistical analysis linking clusters to Grades and Retention; testing Hypotheses H1--H3. \\ \hline
\textbf{May} & \textbf{Predictive Modeling (RQ4) \& Visualization (RQ5):} Training Random Forest models; designing visualizations for the findings. \\ \hline
\textbf{June} & \textbf{Writing \& Revision:} Finalizing the ``Results'' and ``Discussion'' chapters; integrating all chapters; final review with supervisors. \\ \hline
\textbf{July} & \textbf{Submission \& Defense:} Final submission of the dissertation document and preparation for the public defense. \\ \hline
\end{tabular}
\caption{Work plan for the Spring 2026 semester.}
\label{tab:workplan}
\end{table}
\chapter{Literature Review}
\label{chap:litreview}
%% Purpose: Establishes the theoretical foundation. Unlike Chapter 3 (which focuses on Technology/EDM), this chapter focuses on the Educational context (K-12 Online Schools, At-risk students) to justify WHY we need better data analysis.

\section{Overview}
%% Summarizes the chapter structure: from Methodology (how we found papers) to Synthesis (what the papers say).

This chapter describes the methodology used to identify and select the scientific literature relevant to this dissertation.
It explains the type of literature review adopted, the search strategy, the inclusion and exclusion criteria, and the process followed to screen and select the final set of papers that compose the document corpus.
In line with the thesis manifest, the review focuses on research about K--12 online learning and full-time virtual schools in the United States, with particular emphasis on their effectiveness, challenges and implications for students, teachers and school systems.
This corpus provides the conceptual and empirical background needed to frame the later analysis of virtual choreographies of learning in Connections Academy schools.

\section{Type of review}
%% Justifies the methodology: A structured review (systematic principles) but with flexibility for narrative synthesis, suitable for a Master's thesis.

Given the objectives of this dissertation, a structured literature review was conducted, following principles of a systematic literature review while allowing some flexibility that is closer to a narrative synthesis.
The goal of the review is to identify, select and document existing research on K--12 online learning and virtual schools, so as to support the definition of the dissertation research questions and to provide a solid basis for the later State of the Art chapter and for the empirical analysis of student activity data.

The corpus prioritizes contemporary systematic reviews and empirical studies that reflect the current state of K--12 online learning technologies. Instead of early foundational work, the review relies on recent comprehensive analyses, such as the systematic reviews by Martin et al.~\cite{martin_systematic_2020} and Johnson et al.~\cite{online_johnson_2022}.
It also incorporates critical studies on student outcomes, engagement and school-level implementation of K--12 online programmes, specifically those addressing at-risk populations as explored by Beck~\cite{beck_atrisk_2024,beck_prepared_2023} and Toppin and Toppin~\cite{virtual_toppin_2016}, as well as earlier warnings about the maturity of the field~\cite{Barbour2016Virtual}.

\section{Search strategy}
\label{sec:search-strategy}
%% Details the protocol (Keywords, Databases, Tools) to ensure the search is reproducible and transparent.

The literature search was carried out in 2025, in the months leading up to this deliverable.
The primary search engine used was Google Scholar, which offers broad coverage of peer-reviewed publications in education and related fields.
In addition, two discovery tools: Litmaps and ResearchRabbit were used to expand the initial set of articles through citation-based exploration and visualisation of related work.

The search targeted peer-reviewed journal articles, conference papers and book chapters written in English. Given the rapid evolution of educational technology and the impact of the COVID-19 pandemic, the search prioritized studies published between \textbf{2015 and 2025}. This timeframe ensures that the analyzed studies reflect the current technological infrastructure (modern LMS and learning analytics capabilities) and the post-pandemic reality of K-12 online education.

The main search string used in Google Scholar was iteratively refined, but was based on combinations of the following keywords:

\begin{itemize}
    \item \textbf{Core topic terms:} ``K-12 online learning'', ``virtual schools'', ``virtual schooling'', ``online teaching'', ``online school'', ``distance education''.
    \item \textbf{Outcome / focus terms:} ``effectiveness'', ``student achievement'', ``engagement'', ``school choice'', ``best practices'', ``teacher training'', ``at-risk students''.
    \item \textbf{Context terms:} ``K-12'', ``secondary education'', ``United States'', ``high school''.
\end{itemize}

An example of a typical query used in Google Scholar is:

\begin{quote}
``K-12 online learning'' OR ``virtual schools'' OR ``virtual schooling'' AND (effectiveness OR achievement OR engagement OR ``distance education'')
\end{quote}

The initial Google Scholar searches generated a broad set of potentially relevant results.
These references were exported and then imported into Litmaps and ResearchRabbit.
From this initial seed set, both tools were used to:

\begin{itemize}
    \item identify additional papers that frequently cite or are cited by the seed articles;
    \item visualise clusters of research focusing on K--12 virtual schools and online programmes;
    \item surface key works such as large-scale reviews, empirical studies on student outcomes, and research on teacher practices in virtual schools.
\end{itemize}

Through this process, an initial pool of approximately 150 records was assembled before applying the inclusion and exclusion criteria described below.

\section{Inclusion and exclusion criteria}
%% Defines boundaries: helps explain why certain papers (like Higher Ed or Corporate Training) were rejected.

To ensure the relevance and quality of the selected literature, explicit inclusion and exclusion criteria were defined and applied during the screening process.

\subsection{Inclusion criteria}

A study was included in the corpus if it met all of the following criteria:

\begin{itemize}
    \item \textbf{IC1} -- The study focuses on K--12 online learning, virtual schools or virtual schooling (fully online or primarily online programmes).
    \item \textbf{IC2} -- The study is peer-reviewed (journal article, conference paper or book chapter).
    \item \textbf{IC3} -- The study is written in English.
    \item \textbf{IC4} -- The study was published between \textbf{2015 and 2025}.
    \item \textbf{IC5} -- The study addresses at least one of the following aspects: effectiveness or outcomes of virtual schooling (e.g., student achievement, completion, equity), teaching and instructional practices in online K--12 settings, student engagement and experiences, or system-level issues such as school choice, policy and implementation of virtual schools.
\end{itemize}

\subsection{Exclusion criteria}

Studies were excluded if any of the following applied:

\begin{itemize}
    \item \textbf{EC1} -- The main focus is higher education, adult education or corporate training, rather than K--12 schooling.
    \item \textbf{EC2} -- The publication type is a thesis, dissertation, report, blog post, poster, editorial or other non-peer-reviewed document.
    \item \textbf{EC3} -- The study is not available in full text.
    \item \textbf{EC4} -- The study deals with general educational technology or blended learning without a clear focus on fully or primarily online K--12 programmes.
    \item \textbf{EC5} -- The paper is purely theoretical or opinion-based without sufficient connection to K--12 online or virtual schooling.
\end{itemize}

\section{Screening and selection process}
%% Reports the PRISMA flow (numbers of papers kept/rejected) to demonstrate rigor.

The selection process followed three main stages: (i) title and abstract screening, (ii) full-text assessment, and (iii) citation-based expansion and refinement.

\subsection{Stage 1: Title and abstract screening}

After removing duplicates from the combined Google Scholar, Litmaps and ResearchRabbit exports, approximately 150 unique records remained.
Titles and abstracts were screened against the inclusion and exclusion criteria.
At this stage, studies that clearly focused on higher education, general ICT in education or non-online contexts were removed.
This first screening excluded around 90 records, leaving approximately 60 papers for full-text assessment.

\subsection{Stage 2: Full-text assessment}

The full text of the remaining studies was examined in more detail.
Papers that mentioned online learning only superficially, or that did not present results or discussion directly related to K--12 virtual schools, were excluded.
This stage reduced the set to a core group of studies that provide substantial evidence on K--12 online learning. This includes recent systematic reviews~\cite{martin_systematic_2020,online_johnson_2022} and research on student experiences in online K--12 environments, with a particular focus on support for at-risk students as emphasized by Beck~\cite{beck_atrisk_2024,beck_advocate_2020}.
After full-text assessment, around 20 papers remained.

\subsection{Stage 3: Citation-based refinement}

The core set of papers was then used for backward and forward snowballing.
Their reference lists and citation networks were explored using Litmaps and ResearchRabbit.
This step led to the identification of additional key works, including highly cited recent reviews that had not appeared in the initial Google Scholar rankings.
After applying the same inclusion and exclusion criteria to these additional articles and removing new duplicates, the final corpus consisted of \textbf{12 highly relevant papers}.
This set represents the state-of-the-art in the field and includes:

\begin{itemize}
    \item Systematic reviews of K--12 online learning effectiveness and teaching practices, such as those by Martin et al.~\cite{martin_systematic_2020}, Johnson et al.~\cite{online_johnson_2022}, and Barbour and Hodges~\cite{barbour_preparing_2024};
    \item Empirical studies on student engagement, particularly regarding at-risk populations and support structures, including recent work by Beck~\cite{beck_atrisk_2024,beck_prepared_2023,beck_advocate_2020} and Curtis and Werth~\cite{fostering_curtis_2015};
    \item Analyses of policy, school choice and the growth of K--12 online learning, as discussed by Molnar et al.~\cite{molnar_virtual_2023}.
\end{itemize}

The detailed bibliographic information for these papers is provided in the Bibliography at the end of this document.

% -----------------------------------------------------------------------
% THEMATIC SYNTHESIS BEGINS
% Purpose: Defines the "What" - analyzing the actual content of the selected papers.
% -----------------------------------------------------------------------

\section{Effectiveness of K-12 Online Learning}
\label{sec:effectiveness}
%% Synthesizes findings on whether online schools work. Key Takeaway: It's not about the medium, but the support structures.

A central theme in the selected literature is the comparative effectiveness of full-time virtual schools versus traditional brick-and-mortar settings.
Large-scale annual reports, most notably those by the National Education Policy Center (NEPC), consistently highlight significant performance gaps.
Molnar et al.~\cite{molnar_virtual_2023} report that virtual schools generally exhibit lower graduation rates and lower proficiency rates on standardized assessments when compared to national averages.
They argue that the rapid expansion of these schools often outpaces the development of effective regulatory frameworks and instructional quality assurance.

However, recent systematic reviews caution against simple binary comparisons between ``online'' and ``offline'' schooling.
Martin et al.~\cite{martin_systematic_2020} and Johnson et al.~\cite{online_johnson_2022} synthesize two decades of research to suggest that effectiveness is not an inherent property of the medium but is highly dependent on instructional design, teacher preparation, and student support structures.
Johnson et al.~\cite{online_johnson_2022} specifically note that while the \textit{average} performance may be lower, there are specific contexts—such as credit recovery or advanced coursework—where online learning provides essential opportunities that would otherwise be unavailable.
Barbour and Hodges~\cite{barbour_preparing_2024} further emphasize that the ``age of disruptions'' (post-COVID) requires a shift from questioning \textit{if} online learning works to determining \textit{under what conditions} it is effective.

\section{Supporting At-Risk Populations}
\label{sec:at-risk}
%% Crucial context: Explains that virtual schools serve a harder population. This defends why your thesis focuses on "Risk Signals" (H5).

A critical finding in the literature is that the demographic profile of students in full-time virtual schools differs significantly from traditional public schools.
Toppin and Toppin~\cite{virtual_toppin_2016} and Beck~\cite{beck_prepared_2023} observe that virtual schools frequently attract ``at-risk'' students—those who have experienced bullying, medical issues, or academic failure in face-to-face settings.
This selection bias complicates direct performance comparisons, as virtual schools are often serving a population with higher pre-existing needs.

Success for these learners relies heavily on support beyond the screen.
Beck's extensive research~\cite{beck_advocate_2020, beck_atrisk_2024} highlights the pivotal role of the ``on-site facilitator'' or ``advocate''—typically a parent or guardian.
In a study of cyber schools during the pandemic, Beck and Levine~\cite{beck_advocate_2020} found that the level of parental engagement and their ability to structure the student's physical learning environment were stronger predictors of success than many course-level variables.
Beck~\cite{beck_atrisk_2024} further explores parent perceptions, noting that while families value the safety and flexibility of the online environment, they often feel under-equipped to provide the necessary academic and motivational support.

\section{Engagement and Interaction Patterns}
\label{sec:engagement}
%% Identifies the gap: Existing research uses surveys. Your thesis uses LOGS. This section sets up the "Gap Analysis" for Chapter 3.

Student engagement is widely cited as the primary predictor of retention and academic success in online environments.
Curtis and Werth~\cite{fostering_curtis_2015} identify specific strategies to foster engagement, noting that ``transactional distance'' can lead to feelings of isolation.
They argue that successful online schools must intentionally design for both learner-content and learner-teacher interaction.
Kumi-Yeboah et al.~\cite{exploring_kumiyeboah_2018} similarly found that for minority high school students, interaction with instructors was a key factor in promoting a positive academic self-concept.

However, the literature also reveals a methodological gap in \textit{how} engagement is measured.
Most existing studies rely on self-reported surveys or coarse outcome metrics (e.g., course completion).
There is limited research that utilizes fine-grained log data to understand the \textit{temporal patterns} of engagement—what this thesis defines as ``virtual choreographies.''
This gap underscores the need for the data-driven approach proposed in this dissertation, which aims to move beyond static indicators of engagement to uncover the dynamic daily routines that characterize successful learning in virtual schools.

\section{Data management}
%% Description of how you handled the papers (NotebookLM, BibTeX) - good for showing organization.

All selected papers were organised using a combination of reference management and note-taking tools.
Bibliographic information for each article (title, authors, year, venue, DOI and URL) was stored in a Bib\TeX{} file associated with the Overleaf project, ensuring consistency between the reference list and the citations used in this dissertation.
In addition, the PDFs and summaries of the papers were uploaded to a NotebookLM notebook dedicated to the dissertation.
For each paper, short notes were created including:

\begin{itemize}
    \item a brief summary of the research questions, methods and main findings;
    \item tags indicating the main focus (e.g., effectiveness/outcomes, teaching practices, student engagement, policy and school choice);
    \item information about the context (e.g., U.S.\ state or region, subject area, grade levels).
\end{itemize}

This structure transforms the selected articles into a manageable ``stack of documents'' that can be systematically analysed in the next stages of the project and will support the development of the State of the Art chapter.

\section{Limitations and threats to validity}
%% Standard academic disclaimer: acknowledging what might have been missed (e.g., non-English papers).

The review process is subject to several limitations.
First, although Google Scholar provides broad coverage, it does not index all relevant education databases, so some studies may have been missed.
Second, the search was limited to publications in English, which may exclude research on K--12 online learning conducted in other languages and contexts.
Third, the focus on K--12 virtual schools in the United States means that results from other countries were not systematically considered, even when they appeared in the search results.
Finally, despite the use of explicit inclusion and exclusion criteria, subjective judgement was involved when assessing the relevance of titles, abstracts and full texts.
These limitations will be taken into account when interpreting the findings of the selected studies in the subsequent State of the Art chapter.
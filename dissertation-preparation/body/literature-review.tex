\chapter{Literature Review}
\label{chap:litreview}

%% PURPOSE: Establishes the theoretical foundation. Unlike Chapter 3 (Technology), this chapter focuses on the Educational context (K-12 Online Schools, At-risk students).

\section{Overview}
This chapter describes the methodology used to identify and select the scientific literature relevant to this dissertation.
It explains the type of literature review adopted, the search strategy, the inclusion and exclusion criteria, and the process followed to screen and select the final set of papers that compose the document corpus.
In line with the thesis manifest, the review focuses on research about K--12 online learning and full-time virtual schools in the United States, with particular emphasis on their effectiveness, challenges and implications for students, teachers and school systems.
This corpus provides the conceptual and empirical background needed to frame the later analysis of virtual choreographies of learning in Connections Academy schools.

\section{Type of review}
%% METHODOLOGY: Justifying a "Structured Review" allows for systematic rigor without the full resource overhead of a Cochrane-style Systematic Review.
Given the objectives of this dissertation, a structured literature review was conducted, following principles of a systematic literature review while allowing some flexibility that is closer to a narrative synthesis.
The goal of the review is to identify, select and document existing research on K--12 online learning and virtual schools, so as to support the definition of the dissertation research questions and to provide a solid basis for the later State of the Art chapter and for the empirical analysis of student activity data.
The corpus prioritizes contemporary systematic reviews and empirical studies that reflect the current state of K--12 online learning technologies.
Instead of early foundational work, the review relies on recent comprehensive analyses, such as the systematic reviews by Martin et al.~\cite{martin_systematic_2020} and Johnson et al.~\cite{online_johnson_2022}.
It also incorporates critical studies on student outcomes, engagement and school-level implementation of K--12 online programmes, specifically those addressing at-risk populations as explored by Beck~\cite{beck_atrisk_2024,beck_prepared_2023} and Toppin and Toppin~\cite{virtual_toppin_2016}, as well as earlier warnings about the maturity of the field~\cite{Barbour2016Virtual}.

\section{Search strategy}
\label{sec:search-strategy}
The literature search was carried out in 2025, in the months leading up to this deliverable.
%% FEEDBACK INTEGRATION: Added justification for Google Scholar vs. ERIC/Scopus (Prof. Beck's comment).
The primary search engine used was Google Scholar. While discipline-specific databases like ERIC or Scopus are valuable, Google Scholar was selected as the primary source for its broader indexation of interdisciplinary research, capturing relevant studies at the intersection of Computer Science (Educational Data Mining) and Education that might otherwise be fragmented across distinct databases.

In addition, two discovery tools, Litmaps and ResearchRabbit, were used to expand the initial set of articles through citation-based exploration and visualisation of related work.
The search targeted peer-reviewed journal articles, conference papers and book chapters written in English.
Given the rapid evolution of educational technology and the impact of the COVID-19 pandemic, the search prioritized studies published between \textbf{2015 and 2025}.
This timeframe ensures that the analyzed studies reflect the current technological infrastructure (modern LMS and learning analytics capabilities) and the post-pandemic reality of K--12 online education.

The main search string used in Google Scholar was iteratively refined, but was based on combinations of the following keywords:
\begin{itemize}
    \item \textbf{Core topic terms:} ``K-12 online learning'', ``virtual schools'', ``virtual schooling'', ``online teaching'', ``online school'', ``distance education''.
    \item \textbf{Outcome / focus terms:} ``effectiveness'', ``student achievement'', ``engagement'', ``school choice'', ``best practices'', ``teacher training'', ``at-risk students''.
    \item \textbf{Context terms:} ``K-12'', ``secondary education'', ``United States'', ``high school''.
\end{itemize}

An example of a typical query used in Google Scholar is:
\begin{quote}
``K-12 online learning'' OR ``virtual schools'' OR ``virtual schooling'' AND (effectiveness OR achievement OR engagement OR ``distance education'')
\end{quote}

The initial Google Scholar searches generated a broad set of potentially relevant results.
These references were exported and then imported into Litmaps and ResearchRabbit.
From this initial seed set, both tools were used to identify additional papers that frequently cite or are cited by the seed articles, visualise clusters of research, and surface key works.
Through this process, an initial pool of approximately 150 records was assembled before applying the inclusion and exclusion criteria described below.

\section{Inclusion and exclusion criteria}
To ensure the relevance and quality of the selected literature, explicit inclusion and exclusion criteria were defined and applied during the screening process.

\subsection{Inclusion criteria}
A study was included in the corpus if it met all of the following criteria:
\begin{itemize}
    \item \textbf{IC1} -- The study focuses on K--12 online learning, virtual schools or virtual schooling (fully online or primarily online programmes).
    \item \textbf{IC2} -- The study is peer-reviewed (journal article, conference paper or book chapter).
    \item \textbf{IC3} -- The study is written in English.
    \item \textbf{IC4} -- The study was published between \textbf{2015 and 2025}.
    \item \textbf{IC5} -- The study addresses at least one of the following aspects: effectiveness or outcomes of virtual schooling, teaching practices, student engagement, or system-level issues such as school choice and policy.
\end{itemize}

\subsection{Exclusion criteria}
Studies were excluded if any of the following applied:
\begin{itemize}
    \item \textbf{EC1} -- The main focus is higher education, adult education or corporate training.
    \item \textbf{EC2} -- The publication type is a thesis, dissertation, report, blog post, or other non-peer-reviewed document.
    \item \textbf{EC3} -- The study is not available in full text.
    \item \textbf{EC4} -- The study deals with general educational technology or blended learning without a clear focus on fully or primarily online K--12 programmes.
    \item \textbf{EC5} -- The paper is purely theoretical or opinion-based without sufficient connection to K--12 online or virtual schooling.
\end{itemize}

\section{Screening and selection process}
The selection process followed three main stages: (i) title and abstract screening, (ii) full-text assessment, and (iii) citation-based expansion and refinement.

\subsection{Stage 1: Title and abstract screening}
After removing duplicates, approximately 150 unique records remained. Titles and abstracts were screened against the criteria. Studies clearly focusing on higher education or non-online contexts were removed, leaving approximately 60 papers.

\subsection{Stage 2: Full-text assessment}
The full text of the remaining studies was examined. Papers that mentioned online learning only superficially were excluded. This stage reduced the set to a core group of studies providing substantial evidence, including recent systematic reviews~\cite{martin_systematic_2020,online_johnson_2022} and research on at-risk students~\cite{beck_atrisk_2024,beck_advocate_2020}. Around 20 papers remained.

\subsection{Stage 3: Citation-based refinement}
The core set was used for backward and forward snowballing using Litmaps. After applying the same criteria to these additional articles, the final corpus consisted of \textbf{12 highly relevant papers}.

\section{Summary of Selected Studies}
%% FEEDBACK INTEGRATION: Added Table of Studies (Prof. Beck suggestion) to make the chapter feel more "review-like" and less procedural.
To provide a clear overview of the state-of-the-art, Table~\ref{tab:lit_summary} summarizes the key papers selected for the final corpus.

\begin{table}[ht]
\centering
\scriptsize
\begin{tabular}{|p{3cm}|p{2cm}|p{3cm}|p{5cm}|}
\hline
\textbf{Author (Year)} & \textbf{Type} & \textbf{Focus} & \textbf{Key Insight} \\ \hline
Martin et al. (2020) & Review & Effectiveness & Effectiveness depends on design/support, not just the medium. \\ \hline
Johnson et al. (2022) & Review & Teaching Practices & Importance of teacher facilitation in online settings. \\ \hline
Beck (2023, 2024) & Empirical & At-Risk Students & At-risk students perform better with strong ``advocate'' support. \\ \hline
Curtis \& Werth (2015) & Empirical & Engagement & Transactional distance leads to isolation; needs interaction. \\ \hline
Molnar et al. (2023) & Report & Policy & Virtual schools often lag in graduation rates compared to traditional. \\ \hline
Toppin \& Toppin (2016) & Analysis & Demographics & Virtual schools attract students with prior academic/social issues. \\ \hline
\end{tabular}
\caption{Summary of the core literature corpus.}
\label{tab:lit_summary}
\end{table}

%% THEMATIC SYNTHESIS START
%% Purpose: Defines the "What" - analyzing the actual content of the selected papers.

\section{Effectiveness of K-12 Online Learning}
\label{sec:effectiveness}
A central theme in the selected literature is the comparative effectiveness of full-time virtual schools versus traditional brick-and-mortar settings.
Large-scale reports consistently highlight significant performance gaps~\cite{molnar_virtual_2023}, noting lower graduation rates.
However, recent reviews~\cite{martin_systematic_2020, online_johnson_2022} caution against binary comparisons, suggesting effectiveness is dependent on instructional design and support.
Johnson et al.~\cite{online_johnson_2022} note that while average performance may be lower, online learning provides essential opportunities for credit recovery and advanced coursework.

\section{Supporting At-Risk Populations}
\label{sec:at-risk}
%% CONTEXT: This section defends why the thesis focuses on "Risk Signals" (H5) and specific population controls (IEP/Lunch).
A critical finding is that the demographic profile of virtual schools differs significantly from traditional ones.
Toppin and Toppin~\cite{virtual_toppin_2016} and Beck~\cite{beck_prepared_2023} observe that virtual schools frequently attract ``at-risk'' students—those with prior bullying, medical issues, or academic failure.
Success for these learners relies heavily on support beyond the screen.
Beck's research~\cite{beck_advocate_2020, beck_atrisk_2024} highlights the pivotal role of the ``on-site facilitator'' (typically a parent).
Beck and Levine~\cite{beck_advocate_2020} found that parental engagement was a stronger predictor of success than many course-level variables during the pandemic.

\section{Engagement and Interaction Patterns}
\label{sec:engagement}
%% GAP ANALYSIS: Setting up Chapter 3. Literature uses surveys; we will use logs.
Student engagement is cited as the primary predictor of retention.
Curtis and Werth~\cite{fostering_curtis_2015} identify strategies to foster engagement, noting that ``transactional distance'' leads to isolation.
However, a methodological gap exists: most studies rely on surveys.
There is limited research utilizing fine-grained log data to understand \textit{temporal patterns} of engagement.
This gap underscores the need for the data-driven approach proposed in this dissertation.

\section{Limitations and threats to validity}
The review is subject to limitations. First, Google Scholar does not index all databases, potentially missing some studies. Second, the search was limited to English publications. Third, the focus on U.S. virtual schools excludes international contexts. Finally, subjective judgement was involved in screening. These limitations will be considered in the State of the Art analysis.
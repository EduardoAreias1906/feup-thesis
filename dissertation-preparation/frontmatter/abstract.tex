%% abstract.tex: abstract in PT and EN  (FEUP regulations)
%% -------------------------------------------------------
\chapter*{Resumo}
%
As \textit{Connections Academy}, apoiadas pela Pearson Online and Blended Learning, oferecem escolas totalmente online do Jardim-de-Infância ao 12.º ano em 32 estados dos EUA, servindo mais de 100\,000 alunos, muitos dos quais transitaram do ensino presencial devido a problemas anteriores. Relatórios nacionais indicam piores resultados médios em exames padronizados para alunos de escolas online a tempo inteiro, embora estas escolas matriculem mais estudantes em risco, dificultando comparações diretas. Apesar da escala, existe pouca investigação sobre o que os alunos efectivamente fazem nestes ambientes e como os seus padrões de estudo e interação com docentes/colegas se relacionam com o progresso. Esta dissertação analisa dados das 32 escolas usando o enquadramento de \textit{coreografias virtuais} do INESC TEC para identificar sequências recorrentes de ações de aprendizagem online e a sua ligação ao sucesso. Para tal, explora técnicas de IA como \textit{k}-means, t-SNE/UMAP e modelos supervisionados (árvores de decisão, redes neuronais leves, \textit{random forests}) para descobrir \textit{clusters} e trajetórias comportamentais menos óbvias e enriquecer a compreensão da dinâmica de aprendizagem em escolas K–12 totalmente online.

\acresetall


\acresetall %% to reset the acronym usage

\chapter*{Abstract}
%
Connections Academy schools, supported by Pearson Online and Blended Learning, provide fully online K–12 education across 32 U.S. states and currently serve over 100{,}000 students. Many families choose these schools because their children faced specific problems in prior in-person settings. National reports indicate that students enrolled in full-time online schools often perform worse on standardized exams than peers in traditional public schools; however, fully online schools enroll a higher proportion of at-risk students, which complicates fair comparisons.

Despite the scale of this model, there is limited research on what actually happens in these learning environments: Which patterns of study do students follow? How do they engage with teachers and peers? And how are these patterns connected to learning progress? This dissertation addresses these questions by analyzing data from the 32 Connections Academy schools using the \textit{virtual choreographies} framework developed at INESC TEC. The goal is to identify recurring sequences of online learning actions and examine their relevance for student success.

To that end, the work explores Artificial Intelligence techniques to detect and characterize behavior patterns and trajectories, including unsupervised methods such as \textit{k}-means and dimensionality-reduction techniques (t-SNE/UMAP), alongside supervised models such as decision trees, lightweight neural networks, and random forests. These approaches are expected to surface non-obvious configurations of behavior and to enrich our understanding of student learning dynamics in fully online K–12 schools.

\acresetall %% to reset the acronym usage
